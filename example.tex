% Options for packages loaded elsewhere
\PassOptionsToPackage{unicode}{hyperref}
\PassOptionsToPackage{hyphens}{url}
\PassOptionsToPackage{dvipsnames,svgnames,x11names}{xcolor}
%
\documentclass[
  12pt,
]{exam}

\usepackage{amsmath,amssymb}
\usepackage{iftex}
\ifPDFTeX
  \usepackage[T1]{fontenc}
  \usepackage[utf8]{inputenc}
  \usepackage{textcomp} % provide euro and other symbols
\else % if luatex or xetex
  \usepackage{unicode-math}
  \defaultfontfeatures{Scale=MatchLowercase}
  \defaultfontfeatures[\rmfamily]{Ligatures=TeX,Scale=1}
\fi
\usepackage{lmodern}
\ifPDFTeX\else  
    % xetex/luatex font selection
\fi
% Use upquote if available, for straight quotes in verbatim environments
\IfFileExists{upquote.sty}{\usepackage{upquote}}{}
\IfFileExists{microtype.sty}{% use microtype if available
  \usepackage[]{microtype}
  \UseMicrotypeSet[protrusion]{basicmath} % disable protrusion for tt fonts
}{}
\makeatletter
\@ifundefined{KOMAClassName}{% if non-KOMA class
  \IfFileExists{parskip.sty}{%
    \usepackage{parskip}
  }{% else
    \setlength{\parindent}{0pt}
    \setlength{\parskip}{6pt plus 2pt minus 1pt}}
}{% if KOMA class
  \KOMAoptions{parskip=half}}
\makeatother
\usepackage{xcolor}
\usepackage[margin = .6in]{geometry}
\setlength{\emergencystretch}{3em} % prevent overfull lines
\setcounter{secnumdepth}{-\maxdimen} % remove section numbering
% Make \paragraph and \subparagraph free-standing
\ifx\paragraph\undefined\else
  \let\oldparagraph\paragraph
  \renewcommand{\paragraph}[1]{\oldparagraph{#1}\mbox{}}
\fi
\ifx\subparagraph\undefined\else
  \let\oldsubparagraph\subparagraph
  \renewcommand{\subparagraph}[1]{\oldsubparagraph{#1}\mbox{}}
\fi


\providecommand{\tightlist}{%
  \setlength{\itemsep}{0pt}\setlength{\parskip}{0pt}}\usepackage{longtable,booktabs,array}
\usepackage{calc} % for calculating minipage widths
% Correct order of tables after \paragraph or \subparagraph
\usepackage{etoolbox}
\makeatletter
\patchcmd\longtable{\par}{\if@noskipsec\mbox{}\fi\par}{}{}
\makeatother
% Allow footnotes in longtable head/foot
\IfFileExists{footnotehyper.sty}{\usepackage{footnotehyper}}{\usepackage{footnote}}
\makesavenoteenv{longtable}
\usepackage{graphicx}
\makeatletter
\def\maxwidth{\ifdim\Gin@nat@width>\linewidth\linewidth\else\Gin@nat@width\fi}
\def\maxheight{\ifdim\Gin@nat@height>\textheight\textheight\else\Gin@nat@height\fi}
\makeatother
% Scale images if necessary, so that they will not overflow the page
% margins by default, and it is still possible to overwrite the defaults
% using explicit options in \includegraphics[width, height, ...]{}
\setkeys{Gin}{width=\maxwidth,height=\maxheight,keepaspectratio}
% Set default figure placement to htbp
\makeatletter
\def\fps@figure{htbp}
\makeatother

\usepackage{colortbl}
\usepackage{lastpage}
\makeatletter
\@ifpackageloaded{caption}{}{\usepackage{caption}}
\AtBeginDocument{%
\ifdefined\contentsname
  \renewcommand*\contentsname{Table of contents}
\else
  \newcommand\contentsname{Table of contents}
\fi
\ifdefined\listfigurename
  \renewcommand*\listfigurename{List of Figures}
\else
  \newcommand\listfigurename{List of Figures}
\fi
\ifdefined\listtablename
  \renewcommand*\listtablename{List of Tables}
\else
  \newcommand\listtablename{List of Tables}
\fi
\ifdefined\figurename
  \renewcommand*\figurename{Figure}
\else
  \newcommand\figurename{Figure}
\fi
\ifdefined\tablename
  \renewcommand*\tablename{Table}
\else
  \newcommand\tablename{Table}
\fi
}
\@ifpackageloaded{float}{}{\usepackage{float}}
\floatstyle{ruled}
\@ifundefined{c@chapter}{\newfloat{codelisting}{h}{lop}}{\newfloat{codelisting}{h}{lop}[chapter]}
\floatname{codelisting}{Listing}
\newcommand*\listoflistings{\listof{codelisting}{List of Listings}}
\makeatother
\makeatletter
\makeatother
\makeatletter
\@ifpackageloaded{caption}{}{\usepackage{caption}}
\@ifpackageloaded{subcaption}{}{\usepackage{subcaption}}
\makeatother
\ifLuaTeX
  \usepackage{selnolig}  % disable illegal ligatures
\fi
\usepackage{bookmark}

\IfFileExists{xurl.sty}{\usepackage{xurl}}{} % add URL line breaks if available
\urlstyle{same} % disable monospaced font for URLs
\hypersetup{
  pdftitle={Example Exam},
  colorlinks=true,
  linkcolor={blue},
  filecolor={Maroon},
  citecolor={Blue},
  urlcolor={Blue},
  pdfcreator={LaTeX via pandoc}}

\title{Example Exam}
\author{}
\date{}

\begin{document}
\maketitle

% This is an example exam file written in RMarkdown using the examClass package.

% ———- Exam class options ———- %

%\printanswers

% Set the format of the question numbers and titles. ``thequestion'' is the question number and ``thequestiontitle'' is the question title.
\qformat{\textbf{\thequestion. \textit{\thequestiontitle}}\hfill}

% Set the page style to have a header and footer.
\pagestyle{headandfoot}
\footer{}{Page \thepage\ of \pageref*{LastPage}}{}

% Add a grading table. \addpoints \gradetable[v][questions]

% Optional content that is only printed if answers are printed.
\ifprintanswers
% Whatever is inside here will only be printed if answers are printed.
\textbf{Solutions} \fi

% Optional content that is printed only if answers are not printed.
\ifprintanswers \else
% Whatever is inside here will only be printed if answers are not printed.

\begin{center}
Instructions for students:
Here are some instructions.
\end{center}
\fi

\vspace{0.5cm}

\begin{questions}

\titledquestion{"This is a question"}

This is a general question.

\begin{parts}

\part This is a part. \part This part has subparts.

\begin{subparts}

\subpart

This is a subpart. \subpart This subpart has subsubparts!

\begin{subsubparts}

\subsubpart

This is a subsubpart.

\end{subsubparts}

\end{subparts}

\end{parts}

\titledquestion{Course Load}

This question has 2 parts: (a) and (b).

A university in the Midwest has 7000 undergrads. Based on the
registration records, the 7000 undergrads on average take \(\mu=3.5\)
courses in Spring 2024 with a standard deviation of \(\sigma=0.68\). All
of them take at least one course and nobody take over 5 courses in
Spring 2024.

\begin{parts}

\part[6] Calculate the (approximate) probability that a simple random
sample of 100 undergrads from this university take 3.6 courses or more
on average.

\begin{solution}

Let \(\bar{X}\) be the sample mean of the number of courses the selected
\(100\) undergrads take. By CLT,
\[\bar{X} \dot\sim \mbox{N}(\mu = 3.5, \frac{\sigma}{\sqrt{n}} = 0.068)\,.\]
The probability that the sample mean is greater than 3.6 courses can be
calculated as
\[P(\bar{X} > 3.6) \approx P(Z > \frac{3.6-3.5}{0.068}) = P(Z > 1.47) = 1 - P(Z < 1.47) = 1 - 0.9292 = 0.0708\,.\]

\end{solution}

\begin{solution}

Let \(\bar{X}\) be the sample mean of the number of courses the selected
\(100\) undergrads take. By CLT,
\[\bar{X} \dot\sim \mbox{N}(\mu = 3.5, \frac{\sigma}{\sqrt{n}} = 0.068)\,.\]
The probability that the sample mean is greater than 3.6 courses can be
calculated as
\[P(\bar{X} > 3.6) \approx P(Z > \frac{3.6-3.5}{0.068}) = P(Z > 1.47) = 1 - P(Z < 1.47) = 1 - 0.9292 = 0.0708\,.\]

\begin{itemize}
\item 3pts for CLT - 1pt for the mean and 2 pts for the SE. 
\item 1pt for calculating the z-score.
\item 1pt for finding the lower tail probability. Give the point if the student writes down the R code using "pnorm" function instead of checking the normal table.
\item 1pt for subtracting the lower tail probability from 1 and get the correct final answer.
\item No points will be deducted for rounding errors. 
\item When applicable, please give partial credits for the rest of a question if the students get an incorrect answers in the preceding part, provided the answers are consistent with what the students get previously.
\end{itemize}

\end{solution}

\part[4] Please match the 3 histograms below (I, II, III) with the
correct descriptions (A, B, C) of the histograms by filling A, B, C in
the table below.

\begin{center}
\includegraphics[width=0.96\textwidth,height=\textheight]{example_files/figure-pdf/unnamed-chunk-2-1.pdf}
\end{center}

\begin{itemize}\itemsep=2pt
\item (A) sampling distribution of the average number of courses enrolled for 16 randomly selected undergrads
in this University
\item (B) same as (A) but for 200 randomly selected undergrads
\item (C) histogram of the number of courses enrolled for all 7000 undergrads in this University
\end{itemize}

\begin{solution}

C, B, A

\end{solution}

\end{parts}

\begin{minipage}[t]{0.68\linewidth} 

\begin{flushright}
\begin{tabular}{|c|c|}
\hline
Histogram & Description \\
\hline
I & \\
\hline
II & \\
\hline
III & \\
\hline
\end{tabular}
\end{flushright}
\end{minipage}

\newpage

\titledquestion{Cough Syrup}{[}5{]}

A random sample of \(50\) bottles of a particular brand of cough syrup
is selected and the alcohol content (in milligram) of each bottle is
determined. Let \(\mu\) denote the average alcohol content for the
population of all bottles of the brand. Suppose that the resulting 95\%
confidence interval for \(\mu\) is \((7.8, 9.4)\).

Select all the TRUE statements below.

\begin{checkboxes}

\choice 95\% of the bottles of this type of cough syrup have an alcohol
content from 7.8 to 9.4 milligrams. \choice There is a 95\% chance that
\(\mu\) is between 7.8 and 9.4 milligrams. \choice We can be highly
confident that 95\% of all bottles of this type of cough syrup have an
alcohol content that is between 7.8 and 9.4 milligrams.
\CorrectChoice If we repeat the process of selecting a sample of size
\(50\) bottles and then computing the corresponding 95\% confidence
intervals for many times, then 95\% of the resulting intervals will
include \(\mu\). \CorrectChoice If we increase the confidence level to
99\%, the confidence interval will be wider. \choice The 95\% confidence
interval is NOT valid as the population distribution of the alcohol
content may not be normal. \choice If we repeat the process of selecting
a sample of size \(50\) bottles and then computing the average alcohol
content of the selected \(50\) bottles, 95\% of these sample means will
be between \(7.8\) and \(9.4\). \choice If we test the null hypothesis
that the average alcohol content of this type of cough syrup is \(8\)
milligrams, the two-sided p-value will be below \(0.05\) and we will
reject the null hypothesis. \CorrectChoice The margin of error for the
95\% confidence interval (7.8, 9.4) is 0.8. \choice We can reduce the
margin of error by half if we double the sample size.

\end{checkboxes}

\newpage

\end{questions}

\vspace{10mm}



\end{document}
